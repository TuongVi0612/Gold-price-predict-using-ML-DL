Để đánh giá năng lực dự đoán của các mô hình sử dụng trong đồ án, nhóm sử dụng ba phép đo hiệu suất bao gồm: Mean Absolute Percentage Error (MAPE), Root Mean Square Error (RMSE) và Mean Square Error (MSE).

Độ đo MAPE đo lường sai số tuyệt đối trung bình giữa giá trị dự đoán và giá trị thực tế với công thức như sau:
\[
MAPE = \frac{\sum_{i=1}^{n}\left(\frac{abs(y_i - f_i)}{y_i}\right)}{n}
\]

Độ đo MSE tính toán trung bình bình phương của các sai số giữa giá trị thực tế và giá trị dự báo với công thức sau:
\[
MSE = \frac{\sum(f_i - y_i)^2}{N}
\]

Độ đo RMSE đo lường khoảng cách trung bình giữa các giá trị dự đoán và thực tế với công thức như sau:
\[
RMSE = \sqrt{\frac{\sum_{i=1}^{n}(f_i - y_i)^2}{n}}
\]
Trong đó:\\
    \indent\textbullet\ \(f_{i}\) là giá trị dự đoán cho mẫu thứ i.\\
    \indent\textbullet\ \(y_{i}\) là giá trị thực tế cho mẫu thứ i.\\
    \indent\textbullet\ \(N\) là số lượng mẫu.
